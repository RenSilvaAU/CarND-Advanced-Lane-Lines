
% Default to the notebook output style

    


% Inherit from the specified cell style.




    
\documentclass[11pt]{article}

    
    
    \usepackage[T1]{fontenc}
    % Nicer default font (+ math font) than Computer Modern for most use cases
    \usepackage{mathpazo}

    % Basic figure setup, for now with no caption control since it's done
    % automatically by Pandoc (which extracts ![](path) syntax from Markdown).
    \usepackage{graphicx}
    % We will generate all images so they have a width \maxwidth. This means
    % that they will get their normal width if they fit onto the page, but
    % are scaled down if they would overflow the margins.
    \makeatletter
    \def\maxwidth{\ifdim\Gin@nat@width>\linewidth\linewidth
    \else\Gin@nat@width\fi}
    \makeatother
    \let\Oldincludegraphics\includegraphics
    % Set max figure width to be 80% of text width, for now hardcoded.
    \renewcommand{\includegraphics}[1]{\Oldincludegraphics[width=.8\maxwidth]{#1}}
    % Ensure that by default, figures have no caption (until we provide a
    % proper Figure object with a Caption API and a way to capture that
    % in the conversion process - todo).
    \usepackage{caption}
    \DeclareCaptionLabelFormat{nolabel}{}
    \captionsetup{labelformat=nolabel}

    \usepackage{adjustbox} % Used to constrain images to a maximum size 
    \usepackage{xcolor} % Allow colors to be defined
    \usepackage{enumerate} % Needed for markdown enumerations to work
    \usepackage{geometry} % Used to adjust the document margins
    \usepackage{amsmath} % Equations
    \usepackage{amssymb} % Equations
    \usepackage{textcomp} % defines textquotesingle
    % Hack from http://tex.stackexchange.com/a/47451/13684:
    \AtBeginDocument{%
        \def\PYZsq{\textquotesingle}% Upright quotes in Pygmentized code
    }
    \usepackage{upquote} % Upright quotes for verbatim code
    \usepackage{eurosym} % defines \euro
    \usepackage[mathletters]{ucs} % Extended unicode (utf-8) support
    \usepackage[utf8x]{inputenc} % Allow utf-8 characters in the tex document
    \usepackage{fancyvrb} % verbatim replacement that allows latex
    \usepackage{grffile} % extends the file name processing of package graphics 
                         % to support a larger range 
    % The hyperref package gives us a pdf with properly built
    % internal navigation ('pdf bookmarks' for the table of contents,
    % internal cross-reference links, web links for URLs, etc.)
    \usepackage{hyperref}
    \usepackage{longtable} % longtable support required by pandoc >1.10
    \usepackage{booktabs}  % table support for pandoc > 1.12.2
    \usepackage[inline]{enumitem} % IRkernel/repr support (it uses the enumerate* environment)
    \usepackage[normalem]{ulem} % ulem is needed to support strikethroughs (\sout)
                                % normalem makes italics be italics, not underlines
    

    
    
    % Colors for the hyperref package
    \definecolor{urlcolor}{rgb}{0,.145,.698}
    \definecolor{linkcolor}{rgb}{.71,0.21,0.01}
    \definecolor{citecolor}{rgb}{.12,.54,.11}

    % ANSI colors
    \definecolor{ansi-black}{HTML}{3E424D}
    \definecolor{ansi-black-intense}{HTML}{282C36}
    \definecolor{ansi-red}{HTML}{E75C58}
    \definecolor{ansi-red-intense}{HTML}{B22B31}
    \definecolor{ansi-green}{HTML}{00A250}
    \definecolor{ansi-green-intense}{HTML}{007427}
    \definecolor{ansi-yellow}{HTML}{DDB62B}
    \definecolor{ansi-yellow-intense}{HTML}{B27D12}
    \definecolor{ansi-blue}{HTML}{208FFB}
    \definecolor{ansi-blue-intense}{HTML}{0065CA}
    \definecolor{ansi-magenta}{HTML}{D160C4}
    \definecolor{ansi-magenta-intense}{HTML}{A03196}
    \definecolor{ansi-cyan}{HTML}{60C6C8}
    \definecolor{ansi-cyan-intense}{HTML}{258F8F}
    \definecolor{ansi-white}{HTML}{C5C1B4}
    \definecolor{ansi-white-intense}{HTML}{A1A6B2}

    % commands and environments needed by pandoc snippets
    % extracted from the output of `pandoc -s`
    \providecommand{\tightlist}{%
      \setlength{\itemsep}{0pt}\setlength{\parskip}{0pt}}
    \DefineVerbatimEnvironment{Highlighting}{Verbatim}{commandchars=\\\{\}}
    % Add ',fontsize=\small' for more characters per line
    \newenvironment{Shaded}{}{}
    \newcommand{\KeywordTok}[1]{\textcolor[rgb]{0.00,0.44,0.13}{\textbf{{#1}}}}
    \newcommand{\DataTypeTok}[1]{\textcolor[rgb]{0.56,0.13,0.00}{{#1}}}
    \newcommand{\DecValTok}[1]{\textcolor[rgb]{0.25,0.63,0.44}{{#1}}}
    \newcommand{\BaseNTok}[1]{\textcolor[rgb]{0.25,0.63,0.44}{{#1}}}
    \newcommand{\FloatTok}[1]{\textcolor[rgb]{0.25,0.63,0.44}{{#1}}}
    \newcommand{\CharTok}[1]{\textcolor[rgb]{0.25,0.44,0.63}{{#1}}}
    \newcommand{\StringTok}[1]{\textcolor[rgb]{0.25,0.44,0.63}{{#1}}}
    \newcommand{\CommentTok}[1]{\textcolor[rgb]{0.38,0.63,0.69}{\textit{{#1}}}}
    \newcommand{\OtherTok}[1]{\textcolor[rgb]{0.00,0.44,0.13}{{#1}}}
    \newcommand{\AlertTok}[1]{\textcolor[rgb]{1.00,0.00,0.00}{\textbf{{#1}}}}
    \newcommand{\FunctionTok}[1]{\textcolor[rgb]{0.02,0.16,0.49}{{#1}}}
    \newcommand{\RegionMarkerTok}[1]{{#1}}
    \newcommand{\ErrorTok}[1]{\textcolor[rgb]{1.00,0.00,0.00}{\textbf{{#1}}}}
    \newcommand{\NormalTok}[1]{{#1}}
    
    % Additional commands for more recent versions of Pandoc
    \newcommand{\ConstantTok}[1]{\textcolor[rgb]{0.53,0.00,0.00}{{#1}}}
    \newcommand{\SpecialCharTok}[1]{\textcolor[rgb]{0.25,0.44,0.63}{{#1}}}
    \newcommand{\VerbatimStringTok}[1]{\textcolor[rgb]{0.25,0.44,0.63}{{#1}}}
    \newcommand{\SpecialStringTok}[1]{\textcolor[rgb]{0.73,0.40,0.53}{{#1}}}
    \newcommand{\ImportTok}[1]{{#1}}
    \newcommand{\DocumentationTok}[1]{\textcolor[rgb]{0.73,0.13,0.13}{\textit{{#1}}}}
    \newcommand{\AnnotationTok}[1]{\textcolor[rgb]{0.38,0.63,0.69}{\textbf{\textit{{#1}}}}}
    \newcommand{\CommentVarTok}[1]{\textcolor[rgb]{0.38,0.63,0.69}{\textbf{\textit{{#1}}}}}
    \newcommand{\VariableTok}[1]{\textcolor[rgb]{0.10,0.09,0.49}{{#1}}}
    \newcommand{\ControlFlowTok}[1]{\textcolor[rgb]{0.00,0.44,0.13}{\textbf{{#1}}}}
    \newcommand{\OperatorTok}[1]{\textcolor[rgb]{0.40,0.40,0.40}{{#1}}}
    \newcommand{\BuiltInTok}[1]{{#1}}
    \newcommand{\ExtensionTok}[1]{{#1}}
    \newcommand{\PreprocessorTok}[1]{\textcolor[rgb]{0.74,0.48,0.00}{{#1}}}
    \newcommand{\AttributeTok}[1]{\textcolor[rgb]{0.49,0.56,0.16}{{#1}}}
    \newcommand{\InformationTok}[1]{\textcolor[rgb]{0.38,0.63,0.69}{\textbf{\textit{{#1}}}}}
    \newcommand{\WarningTok}[1]{\textcolor[rgb]{0.38,0.63,0.69}{\textbf{\textit{{#1}}}}}
    
    
    % Define a nice break command that doesn't care if a line doesn't already
    % exist.
    \def\br{\hspace*{\fill} \\* }
    % Math Jax compatability definitions
    \def\gt{>}
    \def\lt{<}
    % Document parameters
    \title{Advanced-Lane-Lines-RSilva}
    
    
    

    % Pygments definitions
    
\makeatletter
\def\PY@reset{\let\PY@it=\relax \let\PY@bf=\relax%
    \let\PY@ul=\relax \let\PY@tc=\relax%
    \let\PY@bc=\relax \let\PY@ff=\relax}
\def\PY@tok#1{\csname PY@tok@#1\endcsname}
\def\PY@toks#1+{\ifx\relax#1\empty\else%
    \PY@tok{#1}\expandafter\PY@toks\fi}
\def\PY@do#1{\PY@bc{\PY@tc{\PY@ul{%
    \PY@it{\PY@bf{\PY@ff{#1}}}}}}}
\def\PY#1#2{\PY@reset\PY@toks#1+\relax+\PY@do{#2}}

\expandafter\def\csname PY@tok@o\endcsname{\def\PY@tc##1{\textcolor[rgb]{0.40,0.40,0.40}{##1}}}
\expandafter\def\csname PY@tok@s\endcsname{\def\PY@tc##1{\textcolor[rgb]{0.73,0.13,0.13}{##1}}}
\expandafter\def\csname PY@tok@gu\endcsname{\let\PY@bf=\textbf\def\PY@tc##1{\textcolor[rgb]{0.50,0.00,0.50}{##1}}}
\expandafter\def\csname PY@tok@mh\endcsname{\def\PY@tc##1{\textcolor[rgb]{0.40,0.40,0.40}{##1}}}
\expandafter\def\csname PY@tok@nl\endcsname{\def\PY@tc##1{\textcolor[rgb]{0.63,0.63,0.00}{##1}}}
\expandafter\def\csname PY@tok@mb\endcsname{\def\PY@tc##1{\textcolor[rgb]{0.40,0.40,0.40}{##1}}}
\expandafter\def\csname PY@tok@vg\endcsname{\def\PY@tc##1{\textcolor[rgb]{0.10,0.09,0.49}{##1}}}
\expandafter\def\csname PY@tok@no\endcsname{\def\PY@tc##1{\textcolor[rgb]{0.53,0.00,0.00}{##1}}}
\expandafter\def\csname PY@tok@kp\endcsname{\def\PY@tc##1{\textcolor[rgb]{0.00,0.50,0.00}{##1}}}
\expandafter\def\csname PY@tok@m\endcsname{\def\PY@tc##1{\textcolor[rgb]{0.40,0.40,0.40}{##1}}}
\expandafter\def\csname PY@tok@gt\endcsname{\def\PY@tc##1{\textcolor[rgb]{0.00,0.27,0.87}{##1}}}
\expandafter\def\csname PY@tok@w\endcsname{\def\PY@tc##1{\textcolor[rgb]{0.73,0.73,0.73}{##1}}}
\expandafter\def\csname PY@tok@s1\endcsname{\def\PY@tc##1{\textcolor[rgb]{0.73,0.13,0.13}{##1}}}
\expandafter\def\csname PY@tok@ne\endcsname{\let\PY@bf=\textbf\def\PY@tc##1{\textcolor[rgb]{0.82,0.25,0.23}{##1}}}
\expandafter\def\csname PY@tok@se\endcsname{\let\PY@bf=\textbf\def\PY@tc##1{\textcolor[rgb]{0.73,0.40,0.13}{##1}}}
\expandafter\def\csname PY@tok@gi\endcsname{\def\PY@tc##1{\textcolor[rgb]{0.00,0.63,0.00}{##1}}}
\expandafter\def\csname PY@tok@s2\endcsname{\def\PY@tc##1{\textcolor[rgb]{0.73,0.13,0.13}{##1}}}
\expandafter\def\csname PY@tok@nt\endcsname{\let\PY@bf=\textbf\def\PY@tc##1{\textcolor[rgb]{0.00,0.50,0.00}{##1}}}
\expandafter\def\csname PY@tok@mf\endcsname{\def\PY@tc##1{\textcolor[rgb]{0.40,0.40,0.40}{##1}}}
\expandafter\def\csname PY@tok@k\endcsname{\let\PY@bf=\textbf\def\PY@tc##1{\textcolor[rgb]{0.00,0.50,0.00}{##1}}}
\expandafter\def\csname PY@tok@vc\endcsname{\def\PY@tc##1{\textcolor[rgb]{0.10,0.09,0.49}{##1}}}
\expandafter\def\csname PY@tok@mo\endcsname{\def\PY@tc##1{\textcolor[rgb]{0.40,0.40,0.40}{##1}}}
\expandafter\def\csname PY@tok@gr\endcsname{\def\PY@tc##1{\textcolor[rgb]{1.00,0.00,0.00}{##1}}}
\expandafter\def\csname PY@tok@ch\endcsname{\let\PY@it=\textit\def\PY@tc##1{\textcolor[rgb]{0.25,0.50,0.50}{##1}}}
\expandafter\def\csname PY@tok@sc\endcsname{\def\PY@tc##1{\textcolor[rgb]{0.73,0.13,0.13}{##1}}}
\expandafter\def\csname PY@tok@ni\endcsname{\let\PY@bf=\textbf\def\PY@tc##1{\textcolor[rgb]{0.60,0.60,0.60}{##1}}}
\expandafter\def\csname PY@tok@kt\endcsname{\def\PY@tc##1{\textcolor[rgb]{0.69,0.00,0.25}{##1}}}
\expandafter\def\csname PY@tok@si\endcsname{\let\PY@bf=\textbf\def\PY@tc##1{\textcolor[rgb]{0.73,0.40,0.53}{##1}}}
\expandafter\def\csname PY@tok@sx\endcsname{\def\PY@tc##1{\textcolor[rgb]{0.00,0.50,0.00}{##1}}}
\expandafter\def\csname PY@tok@sr\endcsname{\def\PY@tc##1{\textcolor[rgb]{0.73,0.40,0.53}{##1}}}
\expandafter\def\csname PY@tok@c1\endcsname{\let\PY@it=\textit\def\PY@tc##1{\textcolor[rgb]{0.25,0.50,0.50}{##1}}}
\expandafter\def\csname PY@tok@ow\endcsname{\let\PY@bf=\textbf\def\PY@tc##1{\textcolor[rgb]{0.67,0.13,1.00}{##1}}}
\expandafter\def\csname PY@tok@mi\endcsname{\def\PY@tc##1{\textcolor[rgb]{0.40,0.40,0.40}{##1}}}
\expandafter\def\csname PY@tok@nv\endcsname{\def\PY@tc##1{\textcolor[rgb]{0.10,0.09,0.49}{##1}}}
\expandafter\def\csname PY@tok@kn\endcsname{\let\PY@bf=\textbf\def\PY@tc##1{\textcolor[rgb]{0.00,0.50,0.00}{##1}}}
\expandafter\def\csname PY@tok@ss\endcsname{\def\PY@tc##1{\textcolor[rgb]{0.10,0.09,0.49}{##1}}}
\expandafter\def\csname PY@tok@gh\endcsname{\let\PY@bf=\textbf\def\PY@tc##1{\textcolor[rgb]{0.00,0.00,0.50}{##1}}}
\expandafter\def\csname PY@tok@kd\endcsname{\let\PY@bf=\textbf\def\PY@tc##1{\textcolor[rgb]{0.00,0.50,0.00}{##1}}}
\expandafter\def\csname PY@tok@gs\endcsname{\let\PY@bf=\textbf}
\expandafter\def\csname PY@tok@sd\endcsname{\let\PY@it=\textit\def\PY@tc##1{\textcolor[rgb]{0.73,0.13,0.13}{##1}}}
\expandafter\def\csname PY@tok@sa\endcsname{\def\PY@tc##1{\textcolor[rgb]{0.73,0.13,0.13}{##1}}}
\expandafter\def\csname PY@tok@nd\endcsname{\def\PY@tc##1{\textcolor[rgb]{0.67,0.13,1.00}{##1}}}
\expandafter\def\csname PY@tok@na\endcsname{\def\PY@tc##1{\textcolor[rgb]{0.49,0.56,0.16}{##1}}}
\expandafter\def\csname PY@tok@cm\endcsname{\let\PY@it=\textit\def\PY@tc##1{\textcolor[rgb]{0.25,0.50,0.50}{##1}}}
\expandafter\def\csname PY@tok@cp\endcsname{\def\PY@tc##1{\textcolor[rgb]{0.74,0.48,0.00}{##1}}}
\expandafter\def\csname PY@tok@go\endcsname{\def\PY@tc##1{\textcolor[rgb]{0.53,0.53,0.53}{##1}}}
\expandafter\def\csname PY@tok@cpf\endcsname{\let\PY@it=\textit\def\PY@tc##1{\textcolor[rgb]{0.25,0.50,0.50}{##1}}}
\expandafter\def\csname PY@tok@vm\endcsname{\def\PY@tc##1{\textcolor[rgb]{0.10,0.09,0.49}{##1}}}
\expandafter\def\csname PY@tok@dl\endcsname{\def\PY@tc##1{\textcolor[rgb]{0.73,0.13,0.13}{##1}}}
\expandafter\def\csname PY@tok@gp\endcsname{\let\PY@bf=\textbf\def\PY@tc##1{\textcolor[rgb]{0.00,0.00,0.50}{##1}}}
\expandafter\def\csname PY@tok@ge\endcsname{\let\PY@it=\textit}
\expandafter\def\csname PY@tok@nn\endcsname{\let\PY@bf=\textbf\def\PY@tc##1{\textcolor[rgb]{0.00,0.00,1.00}{##1}}}
\expandafter\def\csname PY@tok@nc\endcsname{\let\PY@bf=\textbf\def\PY@tc##1{\textcolor[rgb]{0.00,0.00,1.00}{##1}}}
\expandafter\def\csname PY@tok@il\endcsname{\def\PY@tc##1{\textcolor[rgb]{0.40,0.40,0.40}{##1}}}
\expandafter\def\csname PY@tok@vi\endcsname{\def\PY@tc##1{\textcolor[rgb]{0.10,0.09,0.49}{##1}}}
\expandafter\def\csname PY@tok@nb\endcsname{\def\PY@tc##1{\textcolor[rgb]{0.00,0.50,0.00}{##1}}}
\expandafter\def\csname PY@tok@kr\endcsname{\let\PY@bf=\textbf\def\PY@tc##1{\textcolor[rgb]{0.00,0.50,0.00}{##1}}}
\expandafter\def\csname PY@tok@err\endcsname{\def\PY@bc##1{\setlength{\fboxsep}{0pt}\fcolorbox[rgb]{1.00,0.00,0.00}{1,1,1}{\strut ##1}}}
\expandafter\def\csname PY@tok@nf\endcsname{\def\PY@tc##1{\textcolor[rgb]{0.00,0.00,1.00}{##1}}}
\expandafter\def\csname PY@tok@bp\endcsname{\def\PY@tc##1{\textcolor[rgb]{0.00,0.50,0.00}{##1}}}
\expandafter\def\csname PY@tok@sb\endcsname{\def\PY@tc##1{\textcolor[rgb]{0.73,0.13,0.13}{##1}}}
\expandafter\def\csname PY@tok@kc\endcsname{\let\PY@bf=\textbf\def\PY@tc##1{\textcolor[rgb]{0.00,0.50,0.00}{##1}}}
\expandafter\def\csname PY@tok@cs\endcsname{\let\PY@it=\textit\def\PY@tc##1{\textcolor[rgb]{0.25,0.50,0.50}{##1}}}
\expandafter\def\csname PY@tok@fm\endcsname{\def\PY@tc##1{\textcolor[rgb]{0.00,0.00,1.00}{##1}}}
\expandafter\def\csname PY@tok@gd\endcsname{\def\PY@tc##1{\textcolor[rgb]{0.63,0.00,0.00}{##1}}}
\expandafter\def\csname PY@tok@sh\endcsname{\def\PY@tc##1{\textcolor[rgb]{0.73,0.13,0.13}{##1}}}
\expandafter\def\csname PY@tok@c\endcsname{\let\PY@it=\textit\def\PY@tc##1{\textcolor[rgb]{0.25,0.50,0.50}{##1}}}

\def\PYZbs{\char`\\}
\def\PYZus{\char`\_}
\def\PYZob{\char`\{}
\def\PYZcb{\char`\}}
\def\PYZca{\char`\^}
\def\PYZam{\char`\&}
\def\PYZlt{\char`\<}
\def\PYZgt{\char`\>}
\def\PYZsh{\char`\#}
\def\PYZpc{\char`\%}
\def\PYZdl{\char`\$}
\def\PYZhy{\char`\-}
\def\PYZsq{\char`\'}
\def\PYZdq{\char`\"}
\def\PYZti{\char`\~}
% for compatibility with earlier versions
\def\PYZat{@}
\def\PYZlb{[}
\def\PYZrb{]}
\makeatother


    % Exact colors from NB
    \definecolor{incolor}{rgb}{0.0, 0.0, 0.5}
    \definecolor{outcolor}{rgb}{0.545, 0.0, 0.0}



    
    % Prevent overflowing lines due to hard-to-break entities
    \sloppy 
    % Setup hyperref package
    \hypersetup{
      breaklinks=true,  % so long urls are correctly broken across lines
      colorlinks=true,
      urlcolor=urlcolor,
      linkcolor=linkcolor,
      citecolor=citecolor,
      }
    % Slightly bigger margins than the latex defaults
    
    \geometry{verbose,tmargin=1in,bmargin=1in,lmargin=1in,rmargin=1in}
    
    

    \begin{document}
    
    
    \maketitle
    
    

    
    \hypertarget{advanced-lane-lines}{%
\section{Advanced Lane Lines}\label{advanced-lane-lines}}

\hypertarget{student-ren-silva}{%
\subsubsection{Student: Ren Silva}\label{student-ren-silva}}

\begin{center}\rule{0.5\linewidth}{\linethickness}\end{center}

    \hypertarget{camera-calibration}{%
\subsubsection{Camera Calibration}\label{camera-calibration}}

\hypertarget{briefly-state-how-you-computed-the-camera-matrix-and-distortion-coefficients.-provide-an-example-of-a-distortion-corrected-calibration-image.}{%
\paragraph{1. Briefly state how you computed the camera matrix and
distortion coefficients. Provide an example of a distortion corrected
calibration
image.}\label{briefly-state-how-you-computed-the-camera-matrix-and-distortion-coefficients.-provide-an-example-of-a-distortion-corrected-calibration-image.}}

The code for this step is contained in method
\texttt{LaneDetector.calibrate\_camera()} in file
\href{./advanced-lane-lines/detect-lanes.py}{./advanced\_lane\_lines/detect\_lanes.py}

I start by preparing \texttt{object\ points}, which will be the (x, y,
z) coordinates of the chessboard corners in the world. Here I am
assuming the chessboard is fixed on the (x, y) plane at z=0, such that
the object points are the same for each calibration image.

Thus, \texttt{objp} is just a replicated array of coordinates, and
\texttt{objpoints} will be appended with a copy of it every time I
successfully detect all chessboard corners in a test image.
\texttt{imgpoints} will be appended with the (x, y) pixel position of
each of the corners in the image plane with each successful chessboard
detection.

I then used the output \texttt{objpoints} and \texttt{imgpoints} to
compute the camera calibration and distortion coefficients using the
\texttt{cv2.calibrateCamera()} function. I applied this distortion
correction to the test image using the \texttt{cv2.undistort()} function
and obtained this result:

\begin{figure}
\centering
\includegraphics{./examples/undistort_output.png}
\caption{alt text}
\end{figure}

The next cell (just below this cell) contains a call to the calibration
method \texttt{LaneDetector.calibrate\_camera()} with the images
provided in folder \href{./camera_cal}{./camera\_cal/} (this part is
marked as \textbf{training})

Then, I have used the coefficients to undistort the same images (marked
as \textbf{testing})

    \begin{Verbatim}[commandchars=\\\{\}]
{\color{incolor}In [{\color{incolor}2}]:} \PY{k+kn}{import} \PY{n+nn}{matplotlib}\PY{n+nn}{.}\PY{n+nn}{pyplot} \PY{k}{as} \PY{n+nn}{plt}
        \PY{k+kn}{import} \PY{n+nn}{matplotlib}\PY{n+nn}{.}\PY{n+nn}{image} \PY{k}{as} \PY{n+nn}{mping}
        \PY{k+kn}{import} \PY{n+nn}{numpy} \PY{k}{as} \PY{n+nn}{np}
        \PY{k+kn}{import} \PY{n+nn}{cv2}
        \PY{k+kn}{from} \PY{n+nn}{glob} \PY{k}{import} \PY{n}{glob}
        
        \PY{o}{\PYZpc{}}\PY{k}{matplotlib} inline
        
        \PY{k+kn}{from} \PY{n+nn}{advanced\PYZus{}lane\PYZus{}lines}\PY{n+nn}{.}\PY{n+nn}{detect\PYZus{}lanes} \PY{k}{import} \PY{n}{LaneDetector}
        
        
        \PY{n}{HLS\PYZus{}THRESH} \PY{o}{=} \PY{p}{(}\PY{l+m+mi}{80}\PY{p}{,} \PY{l+m+mi}{255}\PY{p}{)} 
        \PY{n}{COLOUR\PYZus{}THRESH} \PY{o}{=} \PY{p}{(}\PY{l+m+mi}{150}\PY{p}{,} \PY{l+m+mi}{255}\PY{p}{)}
        
        \PY{n}{lane\PYZus{}detector} \PY{o}{=} \PY{n}{LaneDetector}\PY{p}{(}\PY{n}{hls\PYZus{}thresh}\PY{o}{=}\PY{n}{HLS\PYZus{}THRESH}\PY{p}{,}\PY{n}{colour\PYZus{}thresh}\PY{o}{=}\PY{n}{COLOUR\PYZus{}THRESH}\PY{p}{)}
        
        
        \PY{l+s+sd}{\PYZsq{}\PYZsq{}\PYZsq{}}
        \PY{l+s+sd}{STEP 0 \PYZhy{}\PYZgt{} Camera Calibration \PYZhy{} demo}
        \PY{l+s+sd}{\PYZsq{}\PYZsq{}\PYZsq{}}
        
        \PY{n}{images} \PY{o}{=} \PY{n}{glob}\PY{p}{(}\PY{l+s+s2}{\PYZdq{}}\PY{l+s+s2}{./camera\PYZus{}cal/*}\PY{l+s+s2}{\PYZdq{}}\PY{p}{)}
        
        \PY{n}{lane\PYZus{}detector}\PY{o}{.}\PY{n}{calibrate\PYZus{}camera}\PY{p}{(}\PY{n}{images}\PY{p}{,}\PY{n}{test\PYZus{}files}\PY{o}{=}\PY{n}{images}\PY{p}{,}\PY{n}{verbose}\PY{o}{=}\PY{l+m+mi}{1}\PY{p}{)}
\end{Verbatim}


    \begin{Verbatim}[commandchars=\\\{\}]

Calibrating camera{\ldots}

Training{\ldots}

    \end{Verbatim}

    \begin{center}
    \adjustimage{max size={0.9\linewidth}{0.9\paperheight}}{output_2_1.png}
    \end{center}
    { \hspace*{\fill} \\}
    
    \begin{center}
    \adjustimage{max size={0.9\linewidth}{0.9\paperheight}}{output_2_2.png}
    \end{center}
    { \hspace*{\fill} \\}
    
    \begin{center}
    \adjustimage{max size={0.9\linewidth}{0.9\paperheight}}{output_2_3.png}
    \end{center}
    { \hspace*{\fill} \\}
    
    \begin{center}
    \adjustimage{max size={0.9\linewidth}{0.9\paperheight}}{output_2_4.png}
    \end{center}
    { \hspace*{\fill} \\}
    
    \begin{center}
    \adjustimage{max size={0.9\linewidth}{0.9\paperheight}}{output_2_5.png}
    \end{center}
    { \hspace*{\fill} \\}
    
    \begin{center}
    \adjustimage{max size={0.9\linewidth}{0.9\paperheight}}{output_2_6.png}
    \end{center}
    { \hspace*{\fill} \\}
    
    \begin{center}
    \adjustimage{max size={0.9\linewidth}{0.9\paperheight}}{output_2_7.png}
    \end{center}
    { \hspace*{\fill} \\}
    
    \begin{center}
    \adjustimage{max size={0.9\linewidth}{0.9\paperheight}}{output_2_8.png}
    \end{center}
    { \hspace*{\fill} \\}
    
    \begin{center}
    \adjustimage{max size={0.9\linewidth}{0.9\paperheight}}{output_2_9.png}
    \end{center}
    { \hspace*{\fill} \\}
    
    \begin{center}
    \adjustimage{max size={0.9\linewidth}{0.9\paperheight}}{output_2_10.png}
    \end{center}
    { \hspace*{\fill} \\}
    
    \begin{center}
    \adjustimage{max size={0.9\linewidth}{0.9\paperheight}}{output_2_11.png}
    \end{center}
    { \hspace*{\fill} \\}
    
    \begin{center}
    \adjustimage{max size={0.9\linewidth}{0.9\paperheight}}{output_2_12.png}
    \end{center}
    { \hspace*{\fill} \\}
    
    \begin{center}
    \adjustimage{max size={0.9\linewidth}{0.9\paperheight}}{output_2_13.png}
    \end{center}
    { \hspace*{\fill} \\}
    
    \begin{center}
    \adjustimage{max size={0.9\linewidth}{0.9\paperheight}}{output_2_14.png}
    \end{center}
    { \hspace*{\fill} \\}
    
    \begin{center}
    \adjustimage{max size={0.9\linewidth}{0.9\paperheight}}{output_2_15.png}
    \end{center}
    { \hspace*{\fill} \\}
    
    \begin{center}
    \adjustimage{max size={0.9\linewidth}{0.9\paperheight}}{output_2_16.png}
    \end{center}
    { \hspace*{\fill} \\}
    
    \begin{center}
    \adjustimage{max size={0.9\linewidth}{0.9\paperheight}}{output_2_17.png}
    \end{center}
    { \hspace*{\fill} \\}
    
    \begin{Verbatim}[commandchars=\\\{\}]

Testing{\ldots}

    \end{Verbatim}

    \begin{center}
    \adjustimage{max size={0.9\linewidth}{0.9\paperheight}}{output_2_19.png}
    \end{center}
    { \hspace*{\fill} \\}
    
    \begin{center}
    \adjustimage{max size={0.9\linewidth}{0.9\paperheight}}{output_2_20.png}
    \end{center}
    { \hspace*{\fill} \\}
    
    \begin{center}
    \adjustimage{max size={0.9\linewidth}{0.9\paperheight}}{output_2_21.png}
    \end{center}
    { \hspace*{\fill} \\}
    
    \begin{center}
    \adjustimage{max size={0.9\linewidth}{0.9\paperheight}}{output_2_22.png}
    \end{center}
    { \hspace*{\fill} \\}
    
    \begin{center}
    \adjustimage{max size={0.9\linewidth}{0.9\paperheight}}{output_2_23.png}
    \end{center}
    { \hspace*{\fill} \\}
    
    \begin{center}
    \adjustimage{max size={0.9\linewidth}{0.9\paperheight}}{output_2_24.png}
    \end{center}
    { \hspace*{\fill} \\}
    
    \begin{center}
    \adjustimage{max size={0.9\linewidth}{0.9\paperheight}}{output_2_25.png}
    \end{center}
    { \hspace*{\fill} \\}
    
    \begin{center}
    \adjustimage{max size={0.9\linewidth}{0.9\paperheight}}{output_2_26.png}
    \end{center}
    { \hspace*{\fill} \\}
    
    \begin{center}
    \adjustimage{max size={0.9\linewidth}{0.9\paperheight}}{output_2_27.png}
    \end{center}
    { \hspace*{\fill} \\}
    
    \begin{center}
    \adjustimage{max size={0.9\linewidth}{0.9\paperheight}}{output_2_28.png}
    \end{center}
    { \hspace*{\fill} \\}
    
    \begin{center}
    \adjustimage{max size={0.9\linewidth}{0.9\paperheight}}{output_2_29.png}
    \end{center}
    { \hspace*{\fill} \\}
    
    \begin{center}
    \adjustimage{max size={0.9\linewidth}{0.9\paperheight}}{output_2_30.png}
    \end{center}
    { \hspace*{\fill} \\}
    
    \begin{center}
    \adjustimage{max size={0.9\linewidth}{0.9\paperheight}}{output_2_31.png}
    \end{center}
    { \hspace*{\fill} \\}
    
    \begin{center}
    \adjustimage{max size={0.9\linewidth}{0.9\paperheight}}{output_2_32.png}
    \end{center}
    { \hspace*{\fill} \\}
    
    \begin{center}
    \adjustimage{max size={0.9\linewidth}{0.9\paperheight}}{output_2_33.png}
    \end{center}
    { \hspace*{\fill} \\}
    
    \begin{center}
    \adjustimage{max size={0.9\linewidth}{0.9\paperheight}}{output_2_34.png}
    \end{center}
    { \hspace*{\fill} \\}
    
    \begin{center}
    \adjustimage{max size={0.9\linewidth}{0.9\paperheight}}{output_2_35.png}
    \end{center}
    { \hspace*{\fill} \\}
    
    \begin{center}
    \adjustimage{max size={0.9\linewidth}{0.9\paperheight}}{output_2_36.png}
    \end{center}
    { \hspace*{\fill} \\}
    
    \begin{center}
    \adjustimage{max size={0.9\linewidth}{0.9\paperheight}}{output_2_37.png}
    \end{center}
    { \hspace*{\fill} \\}
    
    \begin{center}
    \adjustimage{max size={0.9\linewidth}{0.9\paperheight}}{output_2_38.png}
    \end{center}
    { \hspace*{\fill} \\}
    
    \begin{Verbatim}[commandchars=\\\{\}]

Finished Calculating calibration coefficients

    \end{Verbatim}

    \begin{center}\rule{0.5\linewidth}{\linethickness}\end{center}

\hypertarget{pipeline-single-images}{%
\subsubsection{Pipeline (single images)}\label{pipeline-single-images}}

\hypertarget{provide-an-example-of-a-distortion-corrected-image.}{%
\paragraph{1. Provide an example of a distortion-corrected
image.}\label{provide-an-example-of-a-distortion-corrected-image.}}

To demonstrate this step, I will describe how I apply the distortion
correction to one of the test images like this one:
\includegraphics{./test_images/test1.jpg}

    \begin{center}\rule{0.5\linewidth}{\linethickness}\end{center}

\hypertarget{step-1}{%
\subsection{Step 1}\label{step-1}}

One way to undistort an image \textbf{would be} simply calling the
method \texttt{cv2.undistort()} passing the \texttt{lane\_detector.mtx}
and \texttt{detect\_lanes.dist} worked out in the previous step.

However, my choice was to create a method in class \texttt{LaneDetector}
in file
\href{./advanced-lane-lines/detect-lanes.py}{./advanced\_lane\_lines/detect\_lanes.py}.
The method \texttt{LaneDetector.undistort()} remembers the coefficients
worked out in the calibration, and work very similarly to
\texttt{cv2.undistort()} - and this is the method that I will be calling
from now on.

The cell below shows the method applied to one test image, from folder
\href{./test_images}{./test\_images/}:

    \begin{Verbatim}[commandchars=\\\{\}]
{\color{incolor}In [{\color{incolor} }]:} \PY{l+s+sd}{\PYZsq{}\PYZsq{}\PYZsq{}}
        \PY{l+s+sd}{STEP 1 \PYZhy{}\PYZgt{} Undistorting image using detect\PYZus{}lanes.mtx, detect\PYZus{}lanes.dist worked out in camera calibration}
        \PY{l+s+sd}{\PYZsq{}\PYZsq{}\PYZsq{}}
        \PY{n}{img} \PY{o}{=} \PY{n}{mping}\PY{o}{.}\PY{n}{imread}\PY{p}{(}\PY{l+s+s1}{\PYZsq{}}\PY{l+s+s1}{./test\PYZus{}images/test1.jpg}\PY{l+s+s1}{\PYZsq{}}\PY{p}{)}
        
        \PY{n}{dst} \PY{o}{=} \PY{n}{lane\PYZus{}detector}\PY{o}{.}\PY{n}{undistort}\PY{p}{(}\PY{n}{img}\PY{p}{)}
        
        \PY{n}{plt}\PY{o}{.}\PY{n}{figure}\PY{p}{(}\PY{n}{figsize}\PY{o}{=}\PY{p}{(}\PY{l+m+mi}{20}\PY{p}{,}\PY{l+m+mi}{10}\PY{p}{)}\PY{p}{)}
        \PY{n}{plt}\PY{o}{.}\PY{n}{imshow}\PY{p}{(}\PY{n}{dst}\PY{p}{)}
        \PY{n}{plt}\PY{o}{.}\PY{n}{show}\PY{p}{(}\PY{p}{)}  
\end{Verbatim}


    \begin{center}\rule{0.5\linewidth}{\linethickness}\end{center}

\hypertarget{describe-how-and-identify-where-in-your-code-you-used-color-transforms-gradients-or-other-methods-to-create-a-thresholded-binary-image.-provide-an-example-of-a-binary-image-result.}{%
\paragraph{2. Describe how (and identify where in your code) you used
color transforms, gradients or other methods to create a thresholded
binary image. Provide an example of a binary image
result.}\label{describe-how-and-identify-where-in-your-code-you-used-color-transforms-gradients-or-other-methods-to-create-a-thresholded-binary-image.-provide-an-example-of-a-binary-image-result.}}

I used a combination of color and gradient thresholds to generate a
binary image. I combined color (green) AND saturation OR a x-sobel
filter.

Below are 5 types of gradients and filters that I tried before settling
for the combination of three of them (\textbf{combined})

Eventually, I created a method called
\texttt{LaneDetector.lane\_binary()} (in file
\href{./advanced-lane-lines/detect-lanes.py}{./advanced\_lane\_lines/detect\_lanes.py}
) - which I plan to use from now on.

Here are my studies in colour transforms before settling for a
combination of them - testing images provided in folder
\href{./test_images}{./test\_images/}

    \begin{Verbatim}[commandchars=\\\{\}]
{\color{incolor}In [{\color{incolor}3}]:} \PY{l+s+sd}{\PYZsq{}\PYZsq{}\PYZsq{}}
        \PY{l+s+sd}{STEP 2 \PYZhy{}\PYZgt{} Demostrate Colour Space Transforms: hls, colour (rgb), sobel, or mag\PYZus{}thresholding}
        \PY{l+s+sd}{\PYZsq{}\PYZsq{}\PYZsq{}}
        \PY{k}{def} \PY{n+nf}{demo\PYZus{}colour\PYZus{}space\PYZus{}transforms}\PY{p}{(}\PY{n}{file\PYZus{}name}\PY{p}{)}\PY{p}{:}
            
            \PY{n}{img} \PY{o}{=} \PY{n}{mping}\PY{o}{.}\PY{n}{imread}\PY{p}{(}\PY{n}{file\PYZus{}name}\PY{p}{)}
        
            \PY{n}{dst} \PY{o}{=} \PY{n}{lane\PYZus{}detector}\PY{o}{.}\PY{n}{undistort}\PY{p}{(}\PY{n}{img}\PY{p}{)}
        
            \PY{c+c1}{\PYZsh{} now do gradient a transform to get a binary}
            \PY{n}{hls\PYZus{}binary} \PY{o}{=} \PY{n}{lane\PYZus{}detector}\PY{o}{.}\PY{n}{hls\PYZus{}select}\PY{p}{(}\PY{n}{dst}\PY{p}{)}
        
            \PY{c+c1}{\PYZsh{} now do colour gradient a transform to get a binary}
            \PY{n}{colour\PYZus{}binary} \PY{o}{=} \PY{n}{lane\PYZus{}detector}\PY{o}{.}\PY{n}{colour\PYZus{}select}\PY{p}{(}\PY{n}{dst}\PY{p}{)}
        
            
        \PY{c+c1}{\PYZsh{}     \PYZsh{} mag thresholding ... trial}
        \PY{c+c1}{\PYZsh{}     mag\PYZus{}binary = lane\PYZus{}detector.mag\PYZus{}thresh(dst, sobel\PYZus{}kernel=3, mag\PYZus{}thresh=(30, 100))   }
         
            \PY{n}{lane\PYZus{}binary} \PY{o}{=} \PY{n}{lane\PYZus{}detector}\PY{o}{.}\PY{n}{lane\PYZus{}binary}\PY{p}{(}\PY{n}{dst}\PY{p}{)}
            
            \PY{n}{sob\PYZus{}binary} \PY{o}{=} \PY{n}{lane\PYZus{}detector}\PY{o}{.}\PY{n}{abs\PYZus{}sobel\PYZus{}thresh}\PY{p}{(}\PY{n}{dst}\PY{p}{,} \PY{n}{orient}\PY{o}{=}\PY{l+s+s1}{\PYZsq{}}\PY{l+s+s1}{x}\PY{l+s+s1}{\PYZsq{}}\PY{p}{,} \PY{n}{thresh\PYZus{}min}\PY{o}{=}\PY{l+m+mi}{30}\PY{p}{,} \PY{n}{thresh\PYZus{}max}\PY{o}{=}\PY{l+m+mi}{100}\PY{p}{)}
        
            \PY{n+nb}{print}\PY{p}{(}\PY{l+s+s1}{\PYZsq{}}\PY{l+s+se}{\PYZbs{}n}\PY{l+s+se}{\PYZbs{}n}\PY{l+s+s1}{\PYZsq{}}\PY{p}{,}\PY{n}{file\PYZus{}name}\PY{p}{)}
            \PY{n}{plt}\PY{o}{.}\PY{n}{figure}\PY{p}{(}\PY{n}{figsize}\PY{o}{=}\PY{p}{(}\PY{l+m+mi}{20}\PY{p}{,}\PY{l+m+mi}{10}\PY{p}{)}\PY{p}{)}
        
            \PY{n}{plt}\PY{o}{.}\PY{n}{subplot}\PY{p}{(}\PY{l+m+mi}{2}\PY{p}{,}\PY{l+m+mi}{3}\PY{p}{,}\PY{l+m+mi}{1}\PY{p}{)}
            \PY{n}{plt}\PY{o}{.}\PY{n}{title}\PY{p}{(}\PY{l+s+s1}{\PYZsq{}}\PY{l+s+s1}{Undistorted}\PY{l+s+s1}{\PYZsq{}}\PY{p}{)}
            \PY{n}{plt}\PY{o}{.}\PY{n}{imshow}\PY{p}{(}\PY{n}{dst}\PY{p}{)}
        
            \PY{n}{plt}\PY{o}{.}\PY{n}{subplot}\PY{p}{(}\PY{l+m+mi}{2}\PY{p}{,}\PY{l+m+mi}{3}\PY{p}{,}\PY{l+m+mi}{2}\PY{p}{)}
            \PY{n}{plt}\PY{o}{.}\PY{n}{title}\PY{p}{(}\PY{l+s+s1}{\PYZsq{}}\PY{l+s+s1}{HLS binary}\PY{l+s+s1}{\PYZsq{}}\PY{p}{)}
            \PY{n}{plt}\PY{o}{.}\PY{n}{imshow}\PY{p}{(}\PY{n}{hls\PYZus{}binary}\PY{p}{,}\PY{n}{cmap}\PY{o}{=}\PY{l+s+s1}{\PYZsq{}}\PY{l+s+s1}{gray}\PY{l+s+s1}{\PYZsq{}}\PY{p}{)}
        
            \PY{n}{plt}\PY{o}{.}\PY{n}{subplot}\PY{p}{(}\PY{l+m+mi}{2}\PY{p}{,}\PY{l+m+mi}{3}\PY{p}{,}\PY{l+m+mi}{3}\PY{p}{)}
            \PY{n}{plt}\PY{o}{.}\PY{n}{title}\PY{p}{(}\PY{l+s+s1}{\PYZsq{}}\PY{l+s+s1}{Colour binary}\PY{l+s+s1}{\PYZsq{}}\PY{p}{)}
            \PY{n}{plt}\PY{o}{.}\PY{n}{imshow}\PY{p}{(}\PY{n}{colour\PYZus{}binary}\PY{p}{,}\PY{n}{cmap}\PY{o}{=}\PY{l+s+s1}{\PYZsq{}}\PY{l+s+s1}{gray}\PY{l+s+s1}{\PYZsq{}}\PY{p}{)}
        
            \PY{n}{plt}\PY{o}{.}\PY{n}{subplot}\PY{p}{(}\PY{l+m+mi}{2}\PY{p}{,}\PY{l+m+mi}{3}\PY{p}{,}\PY{l+m+mi}{4}\PY{p}{)}
            \PY{n}{plt}\PY{o}{.}\PY{n}{title}\PY{p}{(}\PY{l+s+s1}{\PYZsq{}}\PY{l+s+s1}{Sobel x binary}\PY{l+s+s1}{\PYZsq{}}\PY{p}{)}
            \PY{n}{plt}\PY{o}{.}\PY{n}{imshow}\PY{p}{(}\PY{n}{sob\PYZus{}binary}\PY{p}{,}\PY{n}{cmap}\PY{o}{=}\PY{l+s+s1}{\PYZsq{}}\PY{l+s+s1}{gray}\PY{l+s+s1}{\PYZsq{}}\PY{p}{)}
         
            \PY{n}{plt}\PY{o}{.}\PY{n}{subplot}\PY{p}{(}\PY{l+m+mi}{2}\PY{p}{,}\PY{l+m+mi}{3}\PY{p}{,}\PY{l+m+mi}{5}\PY{p}{)}
            \PY{n}{plt}\PY{o}{.}\PY{n}{title}\PY{p}{(}\PY{l+s+s1}{\PYZsq{}}\PY{l+s+s1}{Combined (lane) binary}\PY{l+s+s1}{\PYZsq{}}\PY{p}{)}
            \PY{n}{plt}\PY{o}{.}\PY{n}{imshow}\PY{p}{(}\PY{n}{lane\PYZus{}binary}\PY{p}{,}\PY{n}{cmap}\PY{o}{=}\PY{l+s+s1}{\PYZsq{}}\PY{l+s+s1}{gray}\PY{l+s+s1}{\PYZsq{}}\PY{p}{)}
        
            \PY{c+c1}{\PYZsh{} combining binaries ... this will be the content of method lane\PYZus{}binary()}
            \PY{n}{combined} \PY{o}{=} \PY{n}{np}\PY{o}{.}\PY{n}{zeros\PYZus{}like}\PY{p}{(}\PY{n}{hls\PYZus{}binary}\PY{p}{)}
            \PY{n}{combined}\PY{p}{[}\PY{p}{(}\PY{p}{(}\PY{n}{hls\PYZus{}binary} \PY{o}{==} \PY{l+m+mi}{1}\PY{p}{)} \PY{o}{\PYZam{}} \PY{p}{(}\PY{n}{colour\PYZus{}binary} \PY{o}{==} \PY{l+m+mi}{1}\PY{p}{)} \PY{o}{|} \PY{p}{(}\PY{n}{sob\PYZus{}binary} \PY{o}{==} \PY{l+m+mi}{1}\PY{p}{)}\PY{p}{)}\PY{p}{]} \PY{o}{=} \PY{l+m+mi}{1}
            \PY{n}{plt}\PY{o}{.}\PY{n}{subplot}\PY{p}{(}\PY{l+m+mi}{2}\PY{p}{,}\PY{l+m+mi}{3}\PY{p}{,}\PY{l+m+mi}{6}\PY{p}{)}
            \PY{n}{plt}\PY{o}{.}\PY{n}{title}\PY{p}{(}\PY{l+s+s1}{\PYZsq{}}\PY{l+s+s1}{Combined}\PY{l+s+s1}{\PYZsq{}}\PY{p}{)}
            \PY{n}{plt}\PY{o}{.}\PY{n}{imshow}\PY{p}{(}\PY{n}{combined}\PY{p}{,}\PY{n}{cmap}\PY{o}{=}\PY{l+s+s1}{\PYZsq{}}\PY{l+s+s1}{gray}\PY{l+s+s1}{\PYZsq{}}\PY{p}{)}
            \PY{n}{plt}\PY{o}{.}\PY{n}{show}\PY{p}{(}\PY{p}{)}     
            
        \PY{c+c1}{\PYZsh{} read image}
        \PY{n}{files} \PY{o}{=} \PY{n}{glob}\PY{p}{(}\PY{l+s+s1}{\PYZsq{}}\PY{l+s+s1}{./test\PYZus{}images/*}\PY{l+s+s1}{\PYZsq{}}\PY{p}{)}
        
        \PY{k}{for} \PY{n}{file} \PY{o+ow}{in} \PY{n}{files}\PY{p}{:}
            \PY{n}{demo\PYZus{}colour\PYZus{}space\PYZus{}transforms}\PY{p}{(}\PY{n}{file}\PY{p}{)}
\end{Verbatim}


    \begin{Verbatim}[commandchars=\\\{\}]


 ./test\_images/test6.jpg

    \end{Verbatim}

    \begin{center}
    \adjustimage{max size={0.9\linewidth}{0.9\paperheight}}{output_7_1.png}
    \end{center}
    { \hspace*{\fill} \\}
    
    \begin{Verbatim}[commandchars=\\\{\}]


 ./test\_images/test5.jpg

    \end{Verbatim}

    \begin{center}
    \adjustimage{max size={0.9\linewidth}{0.9\paperheight}}{output_7_3.png}
    \end{center}
    { \hspace*{\fill} \\}
    
    \begin{Verbatim}[commandchars=\\\{\}]


 ./test\_images/test4.jpg

    \end{Verbatim}

    \begin{center}
    \adjustimage{max size={0.9\linewidth}{0.9\paperheight}}{output_7_5.png}
    \end{center}
    { \hspace*{\fill} \\}
    
    \begin{Verbatim}[commandchars=\\\{\}]


 ./test\_images/test1.jpg

    \end{Verbatim}

    \begin{center}
    \adjustimage{max size={0.9\linewidth}{0.9\paperheight}}{output_7_7.png}
    \end{center}
    { \hspace*{\fill} \\}
    
    \begin{Verbatim}[commandchars=\\\{\}]


 ./test\_images/test3.jpg

    \end{Verbatim}

    \begin{center}
    \adjustimage{max size={0.9\linewidth}{0.9\paperheight}}{output_7_9.png}
    \end{center}
    { \hspace*{\fill} \\}
    
    \begin{Verbatim}[commandchars=\\\{\}]


 ./test\_images/test2.jpg

    \end{Verbatim}

    \begin{center}
    \adjustimage{max size={0.9\linewidth}{0.9\paperheight}}{output_7_11.png}
    \end{center}
    { \hspace*{\fill} \\}
    
    \begin{Verbatim}[commandchars=\\\{\}]


 ./test\_images/straight\_lines2.jpg

    \end{Verbatim}

    \begin{center}
    \adjustimage{max size={0.9\linewidth}{0.9\paperheight}}{output_7_13.png}
    \end{center}
    { \hspace*{\fill} \\}
    
    \begin{Verbatim}[commandchars=\\\{\}]


 ./test\_images/straight\_lines1.jpg

    \end{Verbatim}

    \begin{center}
    \adjustimage{max size={0.9\linewidth}{0.9\paperheight}}{output_7_15.png}
    \end{center}
    { \hspace*{\fill} \\}
    
    \begin{center}\rule{0.5\linewidth}{\linethickness}\end{center}

\hypertarget{describe-how-and-identify-where-in-your-code-you-performed-a-perspective-transform-and-provide-an-example-of-a-transformed-image.}{%
\paragraph{3. Describe how (and identify where in your code) you
performed a perspective transform and provide an example of a
transformed
image.}\label{describe-how-and-identify-where-in-your-code-you-performed-a-perspective-transform-and-provide-an-example-of-a-transformed-image.}}

The code for my perspective transform includes a method called
\texttt{LaneDetector.warper()}, which appears in file
\href{./advanced-lane-lines/detect-lanes.py}{./advanced\_lane\_lines/detect\_lanes.py}.

The \texttt{LaneDetector.warper()} method takes as inputs an image
(\texttt{img}), as well as source (\texttt{src}) and destination
(\texttt{dst}) points.

I chose the hardcode the source and destination points in the following
manner:

\begin{Shaded}
\begin{Highlighting}[]
\VariableTok{self}\NormalTok{.src }\OperatorTok{=}\NormalTok{ np.float32(}
\NormalTok{    [[(img_size[}\DecValTok{0}\NormalTok{] }\OperatorTok{/} \DecValTok{2}\NormalTok{) }\OperatorTok{-} \DecValTok{55}\NormalTok{, img_size[}\DecValTok{1}\NormalTok{] }\OperatorTok{/} \DecValTok{2} \OperatorTok{+} \DecValTok{100}\NormalTok{],}
\NormalTok{    [((img_size[}\DecValTok{0}\NormalTok{] }\OperatorTok{/} \DecValTok{6}\NormalTok{) }\OperatorTok{-} \DecValTok{10}\NormalTok{), img_size[}\DecValTok{1}\NormalTok{]],}
\NormalTok{    [(img_size[}\DecValTok{0}\NormalTok{] }\OperatorTok{*} \DecValTok{5} \OperatorTok{/} \DecValTok{6}\NormalTok{) }\OperatorTok{+} \DecValTok{60}\NormalTok{, img_size[}\DecValTok{1}\NormalTok{]],}
\NormalTok{    [(img_size[}\DecValTok{0}\NormalTok{] }\OperatorTok{/} \DecValTok{2} \OperatorTok{+} \DecValTok{55}\NormalTok{), img_size[}\DecValTok{1}\NormalTok{] }\OperatorTok{/} \DecValTok{2} \OperatorTok{+} \DecValTok{100}\NormalTok{]])}
\VariableTok{self}\NormalTok{.dst }\OperatorTok{=}\NormalTok{ np.float32(}
\NormalTok{    [[(img_size[}\DecValTok{0}\NormalTok{] }\OperatorTok{/} \DecValTok{4}\NormalTok{), }\DecValTok{0}\NormalTok{],}
\NormalTok{    [(img_size[}\DecValTok{0}\NormalTok{] }\OperatorTok{/} \DecValTok{4}\NormalTok{), img_size[}\DecValTok{1}\NormalTok{]],}
\NormalTok{    [(img_size[}\DecValTok{0}\NormalTok{] }\OperatorTok{*} \DecValTok{3} \OperatorTok{/} \DecValTok{4}\NormalTok{), img_size[}\DecValTok{1}\NormalTok{]],}
\NormalTok{    [(img_size[}\DecValTok{0}\NormalTok{] }\OperatorTok{*} \DecValTok{3} \OperatorTok{/} \DecValTok{4}\NormalTok{), }\DecValTok{0}\NormalTok{]])}
\end{Highlighting}
\end{Shaded}

This resulted in the following source and destination points:

\begin{longtable}[]{@{}cc@{}}
\toprule
Source & Destination\tabularnewline
\midrule
\endhead
585, 460 & 320, 0\tabularnewline
203, 720 & 320, 720\tabularnewline
1127, 720 & 960, 720\tabularnewline
695, 460 & 960, 0\tabularnewline
\bottomrule
\end{longtable}

I verified that my perspective transform was working as expected by
drawing the \texttt{src} and \texttt{dst} points onto test images and
their warped counterparts, verifying that the lines appear parallel in
the warped images.

This was done all all images of provided in folder
\href{./test_images}{./test\_images/}

    \begin{Verbatim}[commandchars=\\\{\}]
{\color{incolor}In [{\color{incolor}5}]:} \PY{l+s+sd}{\PYZsq{}\PYZsq{}\PYZsq{}}
        \PY{l+s+sd}{STEP 3 \PYZhy{}\PYZgt{} Demostrate how to warp images}
        \PY{l+s+sd}{\PYZsq{}\PYZsq{}\PYZsq{}}
        
        \PY{k}{def} \PY{n+nf}{demo\PYZus{}warper}\PY{p}{(}\PY{n}{file\PYZus{}name}\PY{p}{)}\PY{p}{:}
            
            \PY{n}{img} \PY{o}{=} \PY{n}{mping}\PY{o}{.}\PY{n}{imread}\PY{p}{(}\PY{n}{file\PYZus{}name}\PY{p}{)}
        
            \PY{n}{imgu} \PY{o}{=} \PY{n}{lane\PYZus{}detector}\PY{o}{.}\PY{n}{undistort}\PY{p}{(}\PY{n}{img}\PY{p}{)}
        
            \PY{n}{lane\PYZus{}detector}\PY{o}{.}\PY{n}{set\PYZus{}src\PYZus{}dst}\PY{p}{(}\PY{p}{(}\PY{n}{imgu}\PY{o}{.}\PY{n}{shape}\PY{p}{[}\PY{l+m+mi}{1}\PY{p}{]}\PY{p}{,} \PY{n}{imgu}\PY{o}{.}\PY{n}{shape}\PY{p}{[}\PY{l+m+mi}{0}\PY{p}{]}\PY{p}{)}\PY{p}{)}
        
            \PY{n}{warped} \PY{o}{=} \PY{n}{lane\PYZus{}detector}\PY{o}{.}\PY{n}{warper}\PY{p}{(}\PY{n}{imgu}\PY{p}{)}
        
            \PY{n}{plt}\PY{o}{.}\PY{n}{figure}\PY{p}{(}\PY{n}{figsize}\PY{o}{=}\PY{p}{(}\PY{l+m+mi}{20}\PY{p}{,}\PY{l+m+mi}{10}\PY{p}{)}\PY{p}{)}
        
            \PY{n}{plt}\PY{o}{.}\PY{n}{subplot}\PY{p}{(}\PY{l+m+mi}{1}\PY{p}{,}\PY{l+m+mi}{3}\PY{p}{,}\PY{l+m+mi}{1}\PY{p}{)}
            \PY{n}{plt}\PY{o}{.}\PY{n}{imshow}\PY{p}{(}\PY{n}{imgu}\PY{p}{)}
        
            \PY{n}{plt}\PY{o}{.}\PY{n}{subplot}\PY{p}{(}\PY{l+m+mi}{1}\PY{p}{,}\PY{l+m+mi}{3}\PY{p}{,}\PY{l+m+mi}{2}\PY{p}{)}
            \PY{n}{plt}\PY{o}{.}\PY{n}{imshow}\PY{p}{(}\PY{n}{warped}\PY{p}{)}
            
            \PY{n}{imgu} \PY{o}{=} \PY{n}{lane\PYZus{}detector}\PY{o}{.}\PY{n}{lane\PYZus{}binary}\PY{p}{(}\PY{n}{warped}\PY{p}{)}
        
            \PY{n}{plt}\PY{o}{.}\PY{n}{subplot}\PY{p}{(}\PY{l+m+mi}{1}\PY{p}{,}\PY{l+m+mi}{3}\PY{p}{,}\PY{l+m+mi}{3}\PY{p}{)}
            \PY{n}{plt}\PY{o}{.}\PY{n}{imshow}\PY{p}{(}\PY{n}{imgu}\PY{p}{,}\PY{n}{cmap}\PY{o}{=}\PY{l+s+s1}{\PYZsq{}}\PY{l+s+s1}{gray}\PY{l+s+s1}{\PYZsq{}}\PY{p}{)}
        
            \PY{n}{plt}\PY{o}{.}\PY{n}{show}\PY{p}{(}\PY{p}{)}  
        
        \PY{c+c1}{\PYZsh{} read image}
        \PY{n}{files} \PY{o}{=} \PY{n}{glob}\PY{p}{(}\PY{l+s+s1}{\PYZsq{}}\PY{l+s+s1}{./test\PYZus{}images/*}\PY{l+s+s1}{\PYZsq{}}\PY{p}{)}
        
        \PY{k}{for} \PY{n}{file} \PY{o+ow}{in} \PY{n}{files}\PY{p}{:}
            \PY{n}{demo\PYZus{}warper}\PY{p}{(}\PY{n}{file}\PY{p}{)}
\end{Verbatim}


    \begin{center}
    \adjustimage{max size={0.9\linewidth}{0.9\paperheight}}{output_9_0.png}
    \end{center}
    { \hspace*{\fill} \\}
    
    \begin{center}
    \adjustimage{max size={0.9\linewidth}{0.9\paperheight}}{output_9_1.png}
    \end{center}
    { \hspace*{\fill} \\}
    
    \begin{center}
    \adjustimage{max size={0.9\linewidth}{0.9\paperheight}}{output_9_2.png}
    \end{center}
    { \hspace*{\fill} \\}
    
    \begin{center}
    \adjustimage{max size={0.9\linewidth}{0.9\paperheight}}{output_9_3.png}
    \end{center}
    { \hspace*{\fill} \\}
    
    \begin{center}
    \adjustimage{max size={0.9\linewidth}{0.9\paperheight}}{output_9_4.png}
    \end{center}
    { \hspace*{\fill} \\}
    
    \begin{center}
    \adjustimage{max size={0.9\linewidth}{0.9\paperheight}}{output_9_5.png}
    \end{center}
    { \hspace*{\fill} \\}
    
    \begin{center}
    \adjustimage{max size={0.9\linewidth}{0.9\paperheight}}{output_9_6.png}
    \end{center}
    { \hspace*{\fill} \\}
    
    \begin{center}
    \adjustimage{max size={0.9\linewidth}{0.9\paperheight}}{output_9_7.png}
    \end{center}
    { \hspace*{\fill} \\}
    
    \begin{center}\rule{0.5\linewidth}{\linethickness}\end{center}

\hypertarget{describe-how-and-identify-where-in-your-code-you-identified-lane-line-pixels-and-fit-their-positions-with-a-polynomial}{%
\paragraph{4. Describe how (and identify where in your code) you
identified lane-line pixels and fit their positions with a
polynomial?}\label{describe-how-and-identify-where-in-your-code-you-identified-lane-line-pixels-and-fit-their-positions-with-a-polynomial}}

Then I did some other stuff and fit my lane lines with a 2nd order
polynomial kinda like this:

\begin{figure}
\centering
\includegraphics{./examples/color_fit_lines.jpg}
\caption{alt text}
\end{figure}

    \begin{Verbatim}[commandchars=\\\{\}]
{\color{incolor}In [{\color{incolor}6}]:} \PY{l+s+sd}{\PYZsq{}\PYZsq{}\PYZsq{}}
        \PY{l+s+sd}{STEP 4 \PYZhy{}\PYZgt{} Demostrate how to fit curves and find curvature}
        \PY{l+s+sd}{\PYZsq{}\PYZsq{}\PYZsq{}}
        
        \PY{k}{def} \PY{n+nf}{demo\PYZus{}lane\PYZus{}curv}\PY{p}{(}\PY{n}{file\PYZus{}name}\PY{p}{)}\PY{p}{:}
            
            \PY{n}{img} \PY{o}{=} \PY{n}{mping}\PY{o}{.}\PY{n}{imread}\PY{p}{(}\PY{n}{file\PYZus{}name}\PY{p}{)}
        
            \PY{n}{imgu} \PY{o}{=} \PY{n}{lane\PYZus{}detector}\PY{o}{.}\PY{n}{undistort}\PY{p}{(}\PY{n}{img}\PY{p}{)}
            
            \PY{n}{lane\PYZus{}detector}\PY{o}{.}\PY{n}{set\PYZus{}src\PYZus{}dst}\PY{p}{(}\PY{p}{(}\PY{n}{imgu}\PY{o}{.}\PY{n}{shape}\PY{p}{[}\PY{l+m+mi}{1}\PY{p}{]}\PY{p}{,} \PY{n}{imgu}\PY{o}{.}\PY{n}{shape}\PY{p}{[}\PY{l+m+mi}{0}\PY{p}{]}\PY{p}{)}\PY{p}{)}
        
            \PY{n}{warped} \PY{o}{=} \PY{n}{lane\PYZus{}detector}\PY{o}{.}\PY{n}{warper}\PY{p}{(}\PY{n}{imgu}\PY{p}{)}
        
            \PY{n}{plt}\PY{o}{.}\PY{n}{figure}\PY{p}{(}\PY{n}{figsize}\PY{o}{=}\PY{p}{(}\PY{l+m+mi}{20}\PY{p}{,}\PY{l+m+mi}{4}\PY{p}{)}\PY{p}{)}
        
            \PY{n}{plt}\PY{o}{.}\PY{n}{subplot}\PY{p}{(}\PY{l+m+mi}{1}\PY{p}{,}\PY{l+m+mi}{3}\PY{p}{,}\PY{l+m+mi}{1}\PY{p}{)}
            
            \PY{c+c1}{\PYZsh{} apply image thresholding and colour gradient}
            \PY{n}{imgu} \PY{o}{=} \PY{n}{lane\PYZus{}detector}\PY{o}{.}\PY{n}{lane\PYZus{}binary}\PY{p}{(}\PY{n}{warped}\PY{p}{)}
            
            \PY{n}{plt}\PY{o}{.}\PY{n}{imshow}\PY{p}{(}\PY{n}{imgu}\PY{p}{,}\PY{n}{cmap}\PY{o}{=}\PY{l+s+s1}{\PYZsq{}}\PY{l+s+s1}{gray}\PY{l+s+s1}{\PYZsq{}}\PY{p}{)}    
        
            \PY{n}{plt}\PY{o}{.}\PY{n}{subplot}\PY{p}{(}\PY{l+m+mi}{1}\PY{p}{,}\PY{l+m+mi}{3}\PY{p}{,}\PY{l+m+mi}{2}\PY{p}{)}    
        
            \PY{n}{with\PYZus{}lanes}\PY{p}{,} \PY{n}{left\PYZus{}fit\PYZus{}cr}\PY{p}{,} \PY{n}{right\PYZus{}fit\PYZus{}cr}\PY{p}{,} \PY{n}{offcentre\PYZus{}cr}\PY{p}{,} \PY{n}{position}\PY{p}{,} \PY{n}{ploty}\PY{p}{,} \PY{n}{polig\PYZus{}x}\PY{p}{,} \PY{n}{polig\PYZus{}y} \PY{o}{=} \PY{n}{lane\PYZus{}detector}\PY{o}{.}\PY{n}{fit\PYZus{}polynomial}\PY{p}{(}\PY{n}{imgu}\PY{p}{,}\PY{n}{plot}\PY{o}{=}\PY{k+kc}{True}\PY{p}{)}
        
            \PY{c+c1}{\PYZsh{} Visualize the resulting histogram}
        
        
            \PY{n}{plt}\PY{o}{.}\PY{n}{subplot}\PY{p}{(}\PY{l+m+mi}{1}\PY{p}{,}\PY{l+m+mi}{3}\PY{p}{,}\PY{l+m+mi}{2}\PY{p}{)}
            \PY{n}{plt}\PY{o}{.}\PY{n}{imshow}\PY{p}{(}\PY{n}{with\PYZus{}lanes}\PY{p}{)}
            
        
            \PY{c+c1}{\PYZsh{} Calculate the radius of curvature in meters for both lane lines}
            \PY{n}{left\PYZus{}curverad}\PY{p}{,} \PY{n}{right\PYZus{}curverad} \PY{o}{=} \PY{n}{lane\PYZus{}detector}\PY{o}{.}\PY{n}{measure\PYZus{}curvature\PYZus{}real}\PY{p}{(}\PY{n}{left\PYZus{}fit\PYZus{}cr}\PY{p}{,} \PY{n}{right\PYZus{}fit\PYZus{}cr}\PY{p}{,} \PY{n}{ploty}\PY{p}{)}
            
            \PY{n}{title} \PY{o}{=} \PY{l+s+s1}{\PYZsq{}}\PY{l+s+s1}{Curv(m): left=}\PY{l+s+si}{\PYZob{}0:.0f\PYZcb{}}\PY{l+s+s1}{ right=}\PY{l+s+si}{\PYZob{}1:.0f\PYZcb{}}\PY{l+s+s1}{ mean=}\PY{l+s+si}{\PYZob{}2:.0f\PYZcb{}}\PY{l+s+s1}{\PYZsq{}}\PY{o}{.}\PY{n}{format}\PY{p}{(}\PY{n}{left\PYZus{}curverad}\PY{p}{,} \PY{n}{right\PYZus{}curverad}\PY{p}{,} \PY{p}{(}\PY{n}{left\PYZus{}curverad} \PY{o}{+}  \PY{n}{right\PYZus{}curverad}\PY{p}{)} \PY{o}{/} \PY{l+m+mi}{2} \PY{p}{)}
            
            \PY{n}{plt}\PY{o}{.}\PY{n}{title}\PY{p}{(}\PY{n}{title}\PY{p}{)}
            
            \PY{n}{plt}\PY{o}{.}\PY{n}{show}\PY{p}{(}\PY{p}{)}  
        
        \PY{c+c1}{\PYZsh{} read image}
        \PY{n}{files} \PY{o}{=} \PY{n}{glob}\PY{p}{(}\PY{l+s+s1}{\PYZsq{}}\PY{l+s+s1}{./test\PYZus{}images/*}\PY{l+s+s1}{\PYZsq{}}\PY{p}{)}
        
        \PY{k}{for} \PY{n}{file} \PY{o+ow}{in} \PY{n}{files}\PY{p}{:}
            \PY{n}{demo\PYZus{}lane\PYZus{}curv}\PY{p}{(}\PY{n}{file}\PY{p}{)}
\end{Verbatim}


    \begin{center}
    \adjustimage{max size={0.9\linewidth}{0.9\paperheight}}{output_11_0.png}
    \end{center}
    { \hspace*{\fill} \\}
    
    \begin{center}
    \adjustimage{max size={0.9\linewidth}{0.9\paperheight}}{output_11_1.png}
    \end{center}
    { \hspace*{\fill} \\}
    
    \begin{center}
    \adjustimage{max size={0.9\linewidth}{0.9\paperheight}}{output_11_2.png}
    \end{center}
    { \hspace*{\fill} \\}
    
    \begin{center}
    \adjustimage{max size={0.9\linewidth}{0.9\paperheight}}{output_11_3.png}
    \end{center}
    { \hspace*{\fill} \\}
    
    \begin{center}
    \adjustimage{max size={0.9\linewidth}{0.9\paperheight}}{output_11_4.png}
    \end{center}
    { \hspace*{\fill} \\}
    
    \begin{center}
    \adjustimage{max size={0.9\linewidth}{0.9\paperheight}}{output_11_5.png}
    \end{center}
    { \hspace*{\fill} \\}
    
    \begin{center}
    \adjustimage{max size={0.9\linewidth}{0.9\paperheight}}{output_11_6.png}
    \end{center}
    { \hspace*{\fill} \\}
    
    \begin{center}
    \adjustimage{max size={0.9\linewidth}{0.9\paperheight}}{output_11_7.png}
    \end{center}
    { \hspace*{\fill} \\}
    
    \begin{center}\rule{0.5\linewidth}{\linethickness}\end{center}

\hypertarget{describe-how-and-identify-where-in-your-code-you-calculated-the-radius-of-curvature-of-the-lane-and-the-position-of-the-vehicle-with-respect-to-center.}{%
\paragraph{5. Describe how (and identify where in your code) you
calculated the radius of curvature of the lane and the position of the
vehicle with respect to
center.}\label{describe-how-and-identify-where-in-your-code-you-calculated-the-radius-of-curvature-of-the-lane-and-the-position-of-the-vehicle-with-respect-to-center.}}

I did this in methods \texttt{LaneDetector.measure\_curvature\_real()},
in file
\href{./advanced-lane-lines/detect-lanes.py}{./advanced\_lane\_lines/detect\_lanes.py}.

The method is defined in this tutorial
\href{https://www.intmath.com/applications-differentiation/8-radius-curvature.php}{awesome
tutorial} - basically it is measuring the radius of the approximating
circle to the curve.

It uses this formula:

Here's an example of my output for this step, testing images provided in
in folder \href{./test_images}{./test\_images/}

    \begin{figure}
\centering
\includegraphics{./examples/example_output.jpg}
\caption{alt text}
\end{figure}

\hypertarget{provide-an-example-image-of-your-result-plotted-back-down-onto-the-road-such-that-the-lane-area-is-identified-clearly.}{%
\paragraph{6. Provide an example image of your result plotted back down
onto the road such that the lane area is identified
clearly.}\label{provide-an-example-image-of-your-result-plotted-back-down-onto-the-road-such-that-the-lane-area-is-identified-clearly.}}

I implemented this step in the method \texttt{LaneDetector.map\_lane()},
in file
\href{./advanced-lane-lines/detect-lanes.py}{./advanced\_lane\_lines/detect\_lanes.py}.

Here are examples of my results testing images provided in in folder
\href{./test_images}{./test\_images/}

    \begin{Verbatim}[commandchars=\\\{\}]
{\color{incolor}In [{\color{incolor}7}]:} \PY{l+s+sd}{\PYZsq{}\PYZsq{}\PYZsq{}}
        \PY{l+s+sd}{STEP 6 \PYZhy{}\PYZgt{} Demostrate how to calculate the lane mask}
        \PY{l+s+sd}{\PYZsq{}\PYZsq{}\PYZsq{}}  
        
        
        \PY{k}{def} \PY{n+nf}{demo\PYZus{}draw\PYZus{}lanes}\PY{p}{(}\PY{n}{file\PYZus{}name}\PY{p}{)}\PY{p}{:}
            
            \PY{n}{img} \PY{o}{=} \PY{n}{mping}\PY{o}{.}\PY{n}{imread}\PY{p}{(}\PY{n}{file\PYZus{}name}\PY{p}{)}
        
            \PY{n}{imgu} \PY{o}{=} \PY{n}{lane\PYZus{}detector}\PY{o}{.}\PY{n}{undistort}\PY{p}{(}\PY{n}{img}\PY{p}{)}
            
            \PY{n}{lane\PYZus{}detector}\PY{o}{.}\PY{n}{set\PYZus{}src\PYZus{}dst}\PY{p}{(}\PY{p}{(}\PY{n}{imgu}\PY{o}{.}\PY{n}{shape}\PY{p}{[}\PY{l+m+mi}{1}\PY{p}{]}\PY{p}{,} \PY{n}{imgu}\PY{o}{.}\PY{n}{shape}\PY{p}{[}\PY{l+m+mi}{0}\PY{p}{]}\PY{p}{)}\PY{p}{)}
        
            \PY{n}{warped} \PY{o}{=} \PY{n}{lane\PYZus{}detector}\PY{o}{.}\PY{n}{warper}\PY{p}{(}\PY{n}{imgu}\PY{p}{)}
            
            \PY{c+c1}{\PYZsh{} apply image thresholding and colour gradient}
            \PY{n}{imgu} \PY{o}{=} \PY{n}{lane\PYZus{}detector}\PY{o}{.}\PY{n}{lane\PYZus{}binary}\PY{p}{(}\PY{n}{warped}\PY{p}{)}
            
        
            \PY{n}{with\PYZus{}lanes}\PY{p}{,} \PY{n}{left\PYZus{}fit\PYZus{}cr}\PY{p}{,} \PY{n}{right\PYZus{}fit\PYZus{}cr}\PY{p}{,} \PY{n}{offcentre\PYZus{}cr}\PY{p}{,} \PY{n}{position}\PY{p}{,} \PY{n}{ploty}\PY{p}{,} \PY{n}{polig\PYZus{}x}\PY{p}{,} \PY{n}{polig\PYZus{}y} \PY{o}{=} \PY{n}{lane\PYZus{}detector}\PY{o}{.}\PY{n}{fit\PYZus{}polynomial}\PY{p}{(}\PY{n}{imgu}\PY{p}{)}
            
        
            \PY{c+c1}{\PYZsh{} Calculate the radius of curvature in meters for both lane lines}
            \PY{n}{left\PYZus{}curverad}\PY{p}{,} \PY{n}{right\PYZus{}curverad} \PY{o}{=} \PY{n}{lane\PYZus{}detector}\PY{o}{.}\PY{n}{measure\PYZus{}curvature\PYZus{}real}\PY{p}{(}\PY{n}{left\PYZus{}fit\PYZus{}cr}\PY{p}{,} \PY{n}{right\PYZus{}fit\PYZus{}cr}\PY{p}{,} \PY{n}{ploty}\PY{p}{)}
            
            \PY{n}{messages} \PY{o}{=} \PY{p}{[}\PY{l+s+s1}{\PYZsq{}}\PY{l+s+s1}{Radius of Curvature = }\PY{l+s+si}{\PYZob{}0:.0f\PYZcb{}}\PY{l+s+s1}{(m)}\PY{l+s+s1}{\PYZsq{}}\PY{o}{.}\PY{n}{format}\PY{p}{(}\PY{p}{(}\PY{n}{left\PYZus{}curverad} \PY{o}{+} \PY{n}{right\PYZus{}curverad}\PY{p}{)} \PY{o}{/} \PY{l+m+mi}{2}\PY{p}{)}\PY{p}{,} 
                        \PY{l+s+s1}{\PYZsq{}}\PY{l+s+s1}{Vehicle is }\PY{l+s+si}{\PYZob{}0:.2f\PYZcb{}}\PY{l+s+s1}{m }\PY{l+s+si}{\PYZob{}1:\PYZcb{}}\PY{l+s+s1}{ of centre}\PY{l+s+s1}{\PYZsq{}}\PY{o}{.}\PY{n}{format}\PY{p}{(}\PY{n}{offcentre\PYZus{}cr}\PY{p}{,}\PY{n}{position}\PY{p}{)} \PY{p}{]}
            
            \PY{n}{lm} \PY{o}{=} \PY{n}{lane\PYZus{}detector}\PY{o}{.}\PY{n}{map\PYZus{}lane}\PY{p}{(}\PY{n}{polig\PYZus{}x}\PY{p}{,}\PY{n}{polig\PYZus{}y}\PY{p}{,}\PY{n}{img}\PY{p}{,}\PY{n}{messages}\PY{p}{)}
        
            \PY{n}{plt}\PY{o}{.}\PY{n}{figure}\PY{p}{(}\PY{n}{figsize}\PY{o}{=}\PY{p}{(}\PY{l+m+mi}{20}\PY{p}{,}\PY{l+m+mi}{8}\PY{p}{)}\PY{p}{)}
            
            \PY{n}{plt}\PY{o}{.}\PY{n}{imshow}\PY{p}{(}\PY{n}{lm}\PY{p}{)}
        
            \PY{n}{plt}\PY{o}{.}\PY{n}{show}\PY{p}{(}\PY{p}{)}  
        
        
            
        
        \PY{c+c1}{\PYZsh{} read image}
        \PY{n}{files} \PY{o}{=} \PY{n}{glob}\PY{p}{(}\PY{l+s+s1}{\PYZsq{}}\PY{l+s+s1}{./test\PYZus{}images/*}\PY{l+s+s1}{\PYZsq{}}\PY{p}{)}
        
        \PY{k}{for} \PY{n}{file} \PY{o+ow}{in} \PY{n}{files}\PY{p}{:}
            \PY{n}{demo\PYZus{}draw\PYZus{}lanes}\PY{p}{(}\PY{n}{file}\PY{p}{)}
\end{Verbatim}


    \begin{Verbatim}[commandchars=\\\{\}]
/src/advanced\_lane\_lines/detect\_lanes.py:347: FutureWarning: comparison to `None` will result in an elementwise object comparison in the future.
  if src == None:
/src/advanced\_lane\_lines/detect\_lanes.py:350: FutureWarning: comparison to `None` will result in an elementwise object comparison in the future.
  if dst == None:

    \end{Verbatim}

    \begin{center}
    \adjustimage{max size={0.9\linewidth}{0.9\paperheight}}{output_14_1.png}
    \end{center}
    { \hspace*{\fill} \\}
    
    \begin{center}
    \adjustimage{max size={0.9\linewidth}{0.9\paperheight}}{output_14_2.png}
    \end{center}
    { \hspace*{\fill} \\}
    
    \begin{center}
    \adjustimage{max size={0.9\linewidth}{0.9\paperheight}}{output_14_3.png}
    \end{center}
    { \hspace*{\fill} \\}
    
    \begin{center}
    \adjustimage{max size={0.9\linewidth}{0.9\paperheight}}{output_14_4.png}
    \end{center}
    { \hspace*{\fill} \\}
    
    \begin{center}
    \adjustimage{max size={0.9\linewidth}{0.9\paperheight}}{output_14_5.png}
    \end{center}
    { \hspace*{\fill} \\}
    
    \begin{center}
    \adjustimage{max size={0.9\linewidth}{0.9\paperheight}}{output_14_6.png}
    \end{center}
    { \hspace*{\fill} \\}
    
    \begin{center}
    \adjustimage{max size={0.9\linewidth}{0.9\paperheight}}{output_14_7.png}
    \end{center}
    { \hspace*{\fill} \\}
    
    \begin{center}
    \adjustimage{max size={0.9\linewidth}{0.9\paperheight}}{output_14_8.png}
    \end{center}
    { \hspace*{\fill} \\}
    
    \begin{center}\rule{0.5\linewidth}{\linethickness}\end{center}

\hypertarget{pipeline-video}{%
\subsubsection{Pipeline (video)}\label{pipeline-video}}

\hypertarget{provide-a-link-to-your-final-video-output.-your-pipeline-should-perform-reasonably-well-on-the-entire-project-video-wobbly-lines-are-ok-but-no-catastrophic-failures-that-would-cause-the-car-to-drive-off-the-road.}{%
\paragraph{1. Provide a link to your final video output. Your pipeline
should perform reasonably well on the entire project video (wobbly lines
are ok but no catastrophic failures that would cause the car to drive
off the
road!).}\label{provide-a-link-to-your-final-video-output.-your-pipeline-should-perform-reasonably-well-on-the-entire-project-video-wobbly-lines-are-ok-but-no-catastrophic-failures-that-would-cause-the-car-to-drive-off-the-road.}}

Here is a Section \ref{output-video}

Here is a \href{./output_images/project_video.mp4}{link to my video
result file}

    \begin{Verbatim}[commandchars=\\\{\}]
{\color{incolor}In [{\color{incolor}8}]:} \PY{l+s+sd}{\PYZsq{}\PYZsq{}\PYZsq{}}
        \PY{l+s+sd}{PIPELINE \PYZhy{}\PYZgt{} Demo the entire pipeline in a video}
        \PY{l+s+sd}{\PYZsq{}\PYZsq{}\PYZsq{}}  
        
        \PY{k+kn}{from} \PY{n+nn}{moviepy}\PY{n+nn}{.}\PY{n+nn}{editor} \PY{k}{import} \PY{n}{VideoFileClip}
        \PY{k+kn}{from} \PY{n+nn}{IPython}\PY{n+nn}{.}\PY{n+nn}{display} \PY{k}{import} \PY{n}{HTML}
        
        \PY{k}{def} \PY{n+nf}{process\PYZus{}image}\PY{p}{(}\PY{n}{image}\PY{p}{)}\PY{p}{:}
            \PY{c+c1}{\PYZsh{} NOTE: The output you return should be a color image (3 channel) for processing video below}
            \PY{c+c1}{\PYZsh{} TODO: put your pipeline here,}
            \PY{c+c1}{\PYZsh{} you should return the final output (image where lines are drawn on lanes)}
            \PY{n}{result} \PY{o}{=} \PY{n}{lane\PYZus{}detector}\PY{o}{.}\PY{n}{find\PYZus{}lanes}\PY{p}{(}\PY{n}{image}\PY{p}{)}
            
            \PY{k}{return} \PY{n}{result}
        
        \PY{k}{def} \PY{n+nf}{process\PYZus{}video}\PY{p}{(}\PY{n}{video\PYZus{}file}\PY{p}{)}\PY{p}{:}
            \PY{n}{white\PYZus{}output} \PY{o}{=} \PY{l+s+s1}{\PYZsq{}}\PY{l+s+s1}{./output\PYZus{}images/}\PY{l+s+s1}{\PYZsq{}} \PY{o}{+} \PY{n}{video\PYZus{}file}
            \PY{c+c1}{\PYZsh{}\PYZsh{} To speed up the testing process you may want to try your pipeline on a shorter subclip of the video}
            \PY{c+c1}{\PYZsh{}\PYZsh{} To do so add .subclip(start\PYZus{}second,end\PYZus{}second) to the end of the line below}
            \PY{c+c1}{\PYZsh{}\PYZsh{} Where start\PYZus{}second and end\PYZus{}second are integer values representing the start and end of the subclip}
            \PY{c+c1}{\PYZsh{}\PYZsh{} You may also uncomment the following line for a subclip of the first 5 seconds}
            \PY{c+c1}{\PYZsh{}\PYZsh{}clip1 = VideoFileClip(\PYZdq{}test\PYZus{}videos/solidWhiteRight.mp4\PYZdq{}).subclip(0,5)}
            \PY{n}{clip1} \PY{o}{=} \PY{n}{VideoFileClip}\PY{p}{(}\PY{n}{video\PYZus{}file}\PY{p}{)}
            \PY{n}{white\PYZus{}clip} \PY{o}{=} \PY{n}{clip1}\PY{o}{.}\PY{n}{fl\PYZus{}image}\PY{p}{(}\PY{n}{process\PYZus{}image}\PY{p}{)} \PY{c+c1}{\PYZsh{}NOTE: this function expects color images!!}
            \PY{o}{\PYZpc{}}\PY{k}{time} white\PYZus{}clip.write\PYZus{}videofile(white\PYZus{}output, audio=False)
            
            \PY{k}{return}\PY{p}{(}\PY{n}{white\PYZus{}output}\PY{p}{)}
\end{Verbatim}


    \begin{Verbatim}[commandchars=\\\{\}]
{\color{incolor}In [{\color{incolor}9}]:} \PY{n}{white\PYZus{}output} \PY{o}{=} \PY{n}{process\PYZus{}video}\PY{p}{(}\PY{l+s+s2}{\PYZdq{}}\PY{l+s+s2}{project\PYZus{}video.mp4}\PY{l+s+s2}{\PYZdq{}}\PY{p}{)}
\end{Verbatim}


    \begin{Verbatim}[commandchars=\\\{\}]
[MoviePy] >>>> Building video ./output\_images/project\_video.mp4
[MoviePy] Writing video ./output\_images/project\_video.mp4

    \end{Verbatim}

    \begin{Verbatim}[commandchars=\\\{\}]
100\%|█████████▉| 1260/1261 [01:06<00:00, 18.91it/s]

    \end{Verbatim}

    \begin{Verbatim}[commandchars=\\\{\}]
[MoviePy] Done.
[MoviePy] >>>> Video ready: ./output\_images/project\_video.mp4 

CPU times: user 1min 36s, sys: 3.65 s, total: 1min 40s
Wall time: 1min 6s

    \end{Verbatim}

    \hypertarget{output-video}{%
\subsubsection{Output Video}\label{output-video}}

    \begin{Verbatim}[commandchars=\\\{\}]
{\color{incolor}In [{\color{incolor}10}]:} \PY{n}{HTML}\PY{p}{(}\PY{l+s+s2}{\PYZdq{}\PYZdq{}\PYZdq{}}
         \PY{l+s+s2}{\PYZlt{}video width=}\PY{l+s+s2}{\PYZdq{}}\PY{l+s+s2}{960}\PY{l+s+s2}{\PYZdq{}}\PY{l+s+s2}{ height=}\PY{l+s+s2}{\PYZdq{}}\PY{l+s+s2}{540}\PY{l+s+s2}{\PYZdq{}}\PY{l+s+s2}{ controls\PYZgt{}}
         \PY{l+s+s2}{  \PYZlt{}source src=}\PY{l+s+s2}{\PYZdq{}}\PY{l+s+si}{\PYZob{}0\PYZcb{}}\PY{l+s+s2}{\PYZdq{}}\PY{l+s+s2}{\PYZgt{}}
         \PY{l+s+s2}{\PYZlt{}/video\PYZgt{}}
         \PY{l+s+s2}{\PYZdq{}\PYZdq{}\PYZdq{}}\PY{o}{.}\PY{n}{format}\PY{p}{(}\PY{n}{white\PYZus{}output}\PY{p}{)}\PY{p}{)}
\end{Verbatim}


\begin{Verbatim}[commandchars=\\\{\}]
{\color{outcolor}Out[{\color{outcolor}10}]:} <IPython.core.display.HTML object>
\end{Verbatim}
            
    \begin{Verbatim}[commandchars=\\\{\}]
{\color{incolor}In [{\color{incolor}11}]:} \PY{n}{white\PYZus{}output} \PY{o}{=} \PY{n}{process\PYZus{}video}\PY{p}{(}\PY{l+s+s2}{\PYZdq{}}\PY{l+s+s2}{challenge\PYZus{}video.mp4}\PY{l+s+s2}{\PYZdq{}}\PY{p}{)}
\end{Verbatim}


    \begin{Verbatim}[commandchars=\\\{\}]
[MoviePy] >>>> Building video ./output\_images/challenge\_video.mp4
[MoviePy] Writing video ./output\_images/challenge\_video.mp4

    \end{Verbatim}

    \begin{Verbatim}[commandchars=\\\{\}]
100\%|██████████| 485/485 [00:24<00:00, 20.00it/s]

    \end{Verbatim}

    \begin{Verbatim}[commandchars=\\\{\}]
[MoviePy] Done.
[MoviePy] >>>> Video ready: ./output\_images/challenge\_video.mp4 

CPU times: user 36.5 s, sys: 1.43 s, total: 37.9 s
Wall time: 24.9 s

    \end{Verbatim}

    \begin{Verbatim}[commandchars=\\\{\}]
{\color{incolor}In [{\color{incolor}12}]:} \PY{n}{HTML}\PY{p}{(}\PY{l+s+s2}{\PYZdq{}\PYZdq{}\PYZdq{}}
         \PY{l+s+s2}{\PYZlt{}video width=}\PY{l+s+s2}{\PYZdq{}}\PY{l+s+s2}{960}\PY{l+s+s2}{\PYZdq{}}\PY{l+s+s2}{ height=}\PY{l+s+s2}{\PYZdq{}}\PY{l+s+s2}{540}\PY{l+s+s2}{\PYZdq{}}\PY{l+s+s2}{ controls\PYZgt{}}
         \PY{l+s+s2}{  \PYZlt{}source src=}\PY{l+s+s2}{\PYZdq{}}\PY{l+s+si}{\PYZob{}0\PYZcb{}}\PY{l+s+s2}{\PYZdq{}}\PY{l+s+s2}{\PYZgt{}}
         \PY{l+s+s2}{\PYZlt{}/video\PYZgt{}}
         \PY{l+s+s2}{\PYZdq{}\PYZdq{}\PYZdq{}}\PY{o}{.}\PY{n}{format}\PY{p}{(}\PY{n}{white\PYZus{}output}\PY{p}{)}\PY{p}{)}
\end{Verbatim}


\begin{Verbatim}[commandchars=\\\{\}]
{\color{outcolor}Out[{\color{outcolor}12}]:} <IPython.core.display.HTML object>
\end{Verbatim}
            
    \begin{Verbatim}[commandchars=\\\{\}]
{\color{incolor}In [{\color{incolor}13}]:} \PY{n}{white\PYZus{}output} \PY{o}{=} \PY{n}{process\PYZus{}video}\PY{p}{(}\PY{l+s+s2}{\PYZdq{}}\PY{l+s+s2}{harder\PYZus{}challenge\PYZus{}video.mp4}\PY{l+s+s2}{\PYZdq{}}\PY{p}{)}
\end{Verbatim}


    \begin{Verbatim}[commandchars=\\\{\}]
[MoviePy] >>>> Building video ./output\_images/harder\_challenge\_video.mp4
[MoviePy] Writing video ./output\_images/harder\_challenge\_video.mp4

    \end{Verbatim}

    \begin{Verbatim}[commandchars=\\\{\}]
100\%|█████████▉| 1199/1200 [01:16<00:00, 15.67it/s]

    \end{Verbatim}

    \begin{Verbatim}[commandchars=\\\{\}]
[MoviePy] Done.
[MoviePy] >>>> Video ready: ./output\_images/harder\_challenge\_video.mp4 

CPU times: user 1min 47s, sys: 3.29 s, total: 1min 50s
Wall time: 1min 17s

    \end{Verbatim}

    \begin{Verbatim}[commandchars=\\\{\}]
{\color{incolor}In [{\color{incolor}14}]:} \PY{n}{HTML}\PY{p}{(}\PY{l+s+s2}{\PYZdq{}\PYZdq{}\PYZdq{}}
         \PY{l+s+s2}{\PYZlt{}video width=}\PY{l+s+s2}{\PYZdq{}}\PY{l+s+s2}{960}\PY{l+s+s2}{\PYZdq{}}\PY{l+s+s2}{ height=}\PY{l+s+s2}{\PYZdq{}}\PY{l+s+s2}{540}\PY{l+s+s2}{\PYZdq{}}\PY{l+s+s2}{ controls\PYZgt{}}
         \PY{l+s+s2}{  \PYZlt{}source src=}\PY{l+s+s2}{\PYZdq{}}\PY{l+s+si}{\PYZob{}0\PYZcb{}}\PY{l+s+s2}{\PYZdq{}}\PY{l+s+s2}{\PYZgt{}}
         \PY{l+s+s2}{\PYZlt{}/video\PYZgt{}}
         \PY{l+s+s2}{\PYZdq{}\PYZdq{}\PYZdq{}}\PY{o}{.}\PY{n}{format}\PY{p}{(}\PY{n}{white\PYZus{}output}\PY{p}{)}\PY{p}{)}
\end{Verbatim}


\begin{Verbatim}[commandchars=\\\{\}]
{\color{outcolor}Out[{\color{outcolor}14}]:} <IPython.core.display.HTML object>
\end{Verbatim}
            
    \begin{center}\rule{0.5\linewidth}{\linethickness}\end{center}

\hypertarget{discussion}{%
\subsubsection{Discussion}\label{discussion}}

\hypertarget{briefly-discuss-any-problems-issues-you-faced-in-your-implementation-of-this-project.-where-will-your-pipeline-likely-fail-what-could-you-do-to-make-it-more-robust}{%
\paragraph{1. Briefly discuss any problems / issues you faced in your
implementation of this project. Where will your pipeline likely fail?
What could you do to make it more
robust?}\label{briefly-discuss-any-problems-issues-you-faced-in-your-implementation-of-this-project.-where-will-your-pipeline-likely-fail-what-could-you-do-to-make-it-more-robust}}

The solution to this problem is not simple to implement, has many moving
parts and is prone to error.

The hardest part is to find a reliable method to find the lanes, when
they exist. I ended up with a combination of colour (two colour spaces:
RGB and HLS) and gradient (sobel) - which works better in as the
lighting conditions work, but by no means is a reliable method.

I managed to get a consistent result in video 1 (hence I am submitting
my project), but I can see how unreliable this method can be.

I do think that a deep-learning-based method will yield better results,
provided we have enough videos with various lighting conditions, to
train it.

Video 2 (which I attempted a few times with no success, but that I only
partially succeeded), demonstrating how changing lighting conditions can
lead to errors.

Video 3 is really hard - and my autonomous vehicle would really have a
hard time staying in the lane.

Once the points that make up the lane lines are identifies, the method
is straightforward.

Another problem that I had was the method to change the perspective -
the hardcoded points provided do not produce a good result in all
conditions - and this led to errors in the calculation of the curvature.

Overall, I really enjoyed the challenge - it is good to see how hard it
is to keep a vehicle on track with vision only.

And a final thought: If it is this hard to find lane lines when they do
exist, imagine when they don't - and the latter is very prevalent.

It must be a serious challenge to drive autonomous vehicles where the
roads are not so well mantained, and where lanes can't be seen.


    % Add a bibliography block to the postdoc
    
    
    
    \end{document}
